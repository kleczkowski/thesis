\documentclass[../../praca.tex]{subfiles}

\begin{document}

\chapter{Wprowadzenie}

Powyższa praca odpowiada na pytanie:
\begin{center}
  \textit{Czy i w jaki sposób można napisać algorytm korzystający z mutowalnego stanu
  zachowując szybkość i czytelność tego rozwiązania w oparciu o funkcyjne
  języki programowania?}
\end{center}

Funkcyjne języki programowania są znane z tego, że unikają mutowalności 
na rzecz kopiowania obiektów niemutowalnych, co niewątpliwie jest nie tak optymalne,
jak mutowanie struktur w miejscu. Stąd autor zaimplementował strumieniujący
wariant adaptacyjnego kodowania arytmetycznego z naciskiem na:
\begin{itemize}
  \item zachowanie własności strumieniowania tego algorytmu z wykorzystaniem stałej
    ilości pamięci do przetwarzania;
  \item zapisanie implementacji w sposób przejrzysty i modularny, co zostanie uzyskane
    za pomocą tzw. potoków;
  \item optymalizacją czasową algorytmu, w szczególności zaimplementowana jest
    struktura danych, która realizuję tabelę częstotliwości, w której aktualizacja
    i odpytywanie jest w czasie logarytmicznym;
  \item bezpieczne użycie mutowalności w miejscu --- użyta jest monada \texttt{ST s a},
    która pozwala na mutowanie struktur w miejscu, przy zachowaniu transparentności
    referencyjnej.
\end{itemize}

Praca również sprawdza, czy zaimplementowane rozwiązanie może konkurować z rozwiązaniami
napisanymi w językach imperatywnych uważanych powszechnie za szybkie, tj. C++. 
Ponadto praca opisuje próby związane z innymi językami wybranymi do implementacji projektu.
Ostatecznie projekt, ze względu na różne problemy natury technicznej, został napsiany
w Haskellu.

\section{Struktura pracy}

Praca dzieli się na sześć rozdziałów. W rozdziale drugim autor analizuje postawione
w pracy problemy, które skupiają się wokół realizacji projektu, oraz dokonuje przeglądu
literatury, która dokładniej opisuje dane problemy i ich rozwiązania.

W rozdziale trzecim autor opisuje inne języki programowania, które mogłyby
zostać użyte do implementacji projektu i wśród wybranych dokonuje ich krytyki
wobec wymagań postawionych w rozdziale drugim. Praca podsumowuje, że niektóre
funkcyjne języki programowania nie nadają się do implementacji tego projektu.

W czwartym rozdziale opisana jest struktura projektu i sposób jego kompilacji, testowania
oraz wdrożenia. Również został opisany sposób generacji dokumentacji projektu.

W piątym rozdziale opisane są testy porównawcze między napisanym projektem a próbką 
referencyjną napisaną w C++ oraz przedstawione są wyniki tych testów z krótkim 
podsumowaniem.

W szóstym rozdziale znajduje się krótko sformułowane, właściwe podsumowanie oraz dalsze
rozważania możliwe do podjęcia w tej pracy.

\end{document}
