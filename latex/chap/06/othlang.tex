\documentclass[../praca.tex]{subfiles}

\begin{document}

\chapter{Kodowanie, a inne funkcyjne języki programowania}

Oprócz Haskella istnieją również inne funkcyjne języki programowania, które 
różnią się filozofią oraz zarysem głównych konceptów w nich użytych.

\section{Przegląd i podział języków funkcyjnych}

Niewątpliwie, języki funkcyjne można dzielić względem:
\begin{itemize}
  \item implementowanych strategii ewaluacji wartości (ang. \emph{evaluation strategy});
  \item tego, czy język jest czysty (ang. \emph{purely functional programming language});
  \item użytego systemu typów i jego ekspresji;
  \item kodu wynikowego i środowiska czasu wykonywania dla kompilowanego kodu. 
\end{itemize}

\subsection{Podział względem strategii ewaluacji}

Strategią ewaluacji nazywamy sposób obliczenia wartości argumentów, w momencie gdy
podajemy do funkcji jej argumenty. Wyróżnia się dwie najbardziej znaczące
klasy strategii~\cite{Abelson:SICP}~\cite{Pierce:TPL}:

\begin{itemize}
  \item strategia gorliwa (ang. \emph{eager evaluation} lub \emph{strict evaluation}),
    w której argumenty funkcji są obliczane obligatoryjnie 
    przed jej uruchomieniem;
  \item strategia leniwa (ang. \emph{lazy evaluation} lub \emph{non-strict evaluation}),
    w której argumenty funkcji są obliczane fakultatywnie przed, lub w trakcie jej 
    działania.
\end{itemize}

\subsubsection{Strategie gorliwe}

Do strategii gorliwych należą m. in. strategie:
\begin{itemize}
  \item podawanie przez wartość (ang. \emph{call-by-value}), w którym podawane argumenty
    są obliczane, kopiowane i związane przez zmienne lokalne reprezentujące
    podane argumenty. Ta strategia jest najbardziej popularną strategią, wykorzystywaną
    w językach ogólnego użytku tj. C++, Java, bądź Pascal. Języki funkcyjne, które 
    implementują tę strategię to m.in. OCaml, Scala, Elixir, czy Rust.
    Zaletą tej strategii jest jej prostota implementacji
    w docelowym języku.
  \item podawanie przez referencję (ang. \emph{call-by-reference}), w którym
    argumenty są wskaźnikami do obliczonych argumentów. Rzadko zdarza się,
    by języki kompletnie opierały się na tej strategii, jednakże wykorzystuje
    się ją hybrydowo z \emph{call-by-value}, by uniknąć zbędnych kopii argumentów.
    Jednym ze znanych języków funkcyjnych, które wykorzystują tę strategię jest
    Rust i stanowi to wyjątek w świecie języków funkcyjnych.
\end{itemize}

\subsubsection{Strategie leniwe}

Do strategii leniwych należą m. in. strategie:
\begin{itemize}
  \item podawanie przez nazwę (ang. \emph{call-by-name}), 
    w którym argumenty nie są ewaluowane, lecz
    są bezpośrednio podstawiane, bądź ich konstrukcja jest delegowana
    do funkcji, które konstruują dane wartości (nazywane \emph{thunkiem}).
    Zaletą tej strategii jest
    fakultatywność ewaluacji argumentów. Jeśli dany argument
    prowadzi do niepoprawnego, bądź potencjalnie nieskończonego, 
    obliczenia, może on zostać zignorowany, przy
    czym funkcja może obliczyć wynik bez użycia tego argumentu.
    Jest to strategia implementowana m. in. przez język Scala, wykorzystując
    nularne funkcje jako argumenty. 
  \item podawanie w miarę potrzeby (ang. \emph{call-by-need}); jest to
    wariant podawania przez nazwę wraz ze spamiętywaniem. Ta strategia jest
    wykorzystywana w wypadku, kiedy język jest czystym językiem funkcyjnym
    i stąd podczas konstrukcji wartości nie może dojść do ewentualnego skonstruowania
    obiektu w zupełny inny sposób niż dotychczas. W częściowy sposób ta
    strategia jest implementowana w języku Scala oraz w zupełności w języku
    Haskelli, bądź Miranda.
\end{itemize}

\subsection{Podział względem czystości języka}

Funkcję \( f \) nazywamy czystą, jeżeli spełnia następujące kryteria:~\cite{Milewski:PF}
\begin{itemize}
  \item \( f \) dla tych samych argumentów zwraca te same wartości
    (postulat transparentności referencyjnej);
  \item ewaluacja \( f \) nie prowadzi do żadnych efektów ubocznych.
\end{itemize}

Funkcje czyste stanowią analogiczną konstrukcje w świecie obliczeń, jaką stanowią
funkcje matematyczne.

\begin{remark}
  Kiedy rozważamy wykorzystanie monad, nie należy mylić ewaluacji funkcji
  dającej nam pewien obiekt monady z jej wykonaniem, ponieważ
  obiekt monady jest konstruowany zgodnie z postulatem transparentności referencyjnej,
  niekoniecznie wykonanie tego obiektu monady musi prowadzić do tych samych wyników.
\end{remark}

Języki czysto funkcyjne to takie języki, w których dowolna funkcja jest czysta. 
Takimi językami są m. in. Haskell, Agda, czy Coq.

\subsection{Podział względem użytego systemu typów}

System typów to zbiór zasad, które nadają pewnym wartością pewną etykietę
nazywaną typem. Typy pozwalają na sformalizowaną weryfikację poprawności
programu względem rodzaju danych --- każdy typ definiuje jakiś szczególny
rodzaj danych i jest utoższamiany ze zbiorem konstruowalnych wartości.~\cite{Pierce:TPL}

Wyróżniamy następujące kryteria w podziale systemach typów:

\begin{itemize}
  \item Czy system typów wymaga specyfikowanie typu w czasie kompilacji,
    lub w czasie wykonywania (statyczny vs. dynamiczny system typów)?
  \item Czy system typów wymusza jawne określanie typów zmiennych
    bądź funkcji, lub posiada system dedukcji typów, który pozwala
    nadać typy bytom bez udziału programisty (typowanie jawne vs.
    typowanie niejawne)?
  \item Czy system typów dopuszcza do niejawnych konwersji, promocji
    typów, lub kompletnie zabrania takich działań (typowanie słabe
    vs. typowanie silne)?
  \item Czy system typów pozwala na definiowanie typów zależnych
    od innych typów, innymi słowy, typów polimorficznych (polimorficzny
    system typów)?
  \item Czy system typów pozwala na definiowanie typów zależnych
    od wartości innych typów (zależne systemy typów)?
\end{itemize}

\subsubsection{Przykłady systemów typów}

W tabeli~\ref{tab:type-systems} zestawiono kilka wybranych języków programowania.

\begin{table}
  \begin{tabular}{|l|l|l|l|l|l|}
    \hline
    Język        & Statyczny/dynamiczny?     & Silny/słaby?     & Typowanie (nie)jawne? & Polimorficzny?     & Zależny?   \\ \hline \hline
    Haskell      & statyczny                 & silny            & niejawne              & tak                & częściowo tak  \\ \hline
    OCaml        & statyczny                 & silny            & niejawne              & tak                & nie            \\ \hline
    Scala        & statyczny                 & silny            & niejawne              & tak                & nie            \\ \hline
    Elixir       & dynamiczny                & silny            & n.d.                  & n.d.               & nie            \\ \hline
    Clojure      & dynamiczny                & silny            & n.d.                  & n.d.               & nie            \\ \hline
    Agda         & statyczny                 & silny            & niejawne              & tak                & tak            \\ \hline
    Coq          & statyczny                 & silny            & niejawne              & tak                & tak            \\ \hline
  \end{tabular}
  \caption{Przegląd  cech systemów typów w wybranych językach programowania}
  \label{tab:type-systems}
\end{table}

\begin{remark}
  Haskell jest częściowo zależnym systemem typów, ponieważ pozwala na parametryzację typów za pomocą
  wartości pochodzących z ograniczonej rodziny typów --- używa się wtedy dyrektywy \texttt{DataKinds}.
\end{remark}

\section{Wybór języków programowania}

Przy próbie napisania projektów w oparciu o inne języki programowania wybrano
następujące języki programowania: Scala oraz Elixir. 

\subsection{Scala}

Scala jest językiem wieloparadygmatowym, opracowanym przez 
Martina Oderskiego~\cite{Odersky:BHS}. Scala łączy programowanie
obiektowe z programowaniem funkcyjnym. Jest również językiem,
w którym przeważającą strategią ewaluacji jest~\emph{call-by-value},
ale również implementuje w pełni~\emph{call-by-name} oraz częściowo
\emph{call-by-need}. Scalę również wyróżnia statyczny, silny i polimorficzny system typów,
który umożliwia parametryzowanie typów za pomocą typów wyższego rodzaju
(ang. \emph{higher-kinded types}) oraz pozwala na implementację
wzorca klas typów (ang. \emph{type-class pattern}) za pomocą
argumentów niejawnych (ang. \emph{implicit arguments}). Ponadto
system typów pozwala na niejawne typowanie. Użyty system typów
jest wariantem Systemu-\(\mathcal{F}_{:>}\).

\subsection{Elixir}

Elixir jest językiem, który łączy cechy programowania funkcyjnego
oraz programowania współbieżnego, stworzonym przez Jos{\' e} Valima
w 2012 roku. Elixir jest uruchamiany na maszynie wirtualnej Erlanga,
która daje możliwość tworzenia wielu procesów bardzo małym 
kosztem~\cite{Juric:EiA}.

Elixir posiada dynamiczny i silny system typów, przez co system typów
nie ma potrzeby implementacji typów polimorficznych.

Dalsze sekcje będą poświęcone krytyce tych języków wobec problemu implementacji
kodowania arytmetycznego w tych językach.

\section{Krytyka wybranych języków funkcyjnych}

\subsection{Kwestia czystości języków}

Scala, jak i Elixir, nie są czystymi językami funkcyjnymi, ponieważ
same pozwalają na dokonywanie pewnych efektów ubocznych. W Scali
można jawnie wywoływać polecenia, jak i definiować mutowalne zmienne
(za pomocą \mintinline{scala}{var}). W Elixirze można w jawny sposób
korzystać z wejścia/wyjścia, czy też wysyłać wiadomości do aktorów.

Brak tego obostrzenia prowadzi do osłabienia znaczenia monad, ponieważ
można modelować efekty uboczne wprost, bez ich użycia. Stąd Scala i Elixir
potrzebują innego modelu mutowalności oraz obliczeń, w przeciwieństwie
do przedstawionego w poprzednim rozdziale. Co więcej, sposób
w jakim można osiągać efekty uboczne nie wiele różni się
od tego, jak osiągane są efekty uboczne w językach imperatywnych.

Stąd dalej będą rozważane próby i ograniczenia języków w związku 
z wprowadzaniem rozwiązań pozwalające na osiągnięcie implementacji
zgodnej z filozofią programowania funkcyjnego.

\subsection{Scala a mutowalny stan}

Zgodnie z poprzednią obserwacją, wykorzystanie zaimplementowanej 
monady stanu w takim kształcie,
w jakim występuje w Haskellu, może przynieść podobne problemy wydajnościowe,
które były rozważane w poprzednim rozdziale, stąd powstaje pytanie, w jaki sposób
osiągnąć bezpośrednią mutowalność w sposób monadyczny.

Pomysłem możliwym do implementacji byłoby użycie monady \texttt{Reader}, która
dawałaby możliwość podejrzenia struktury, która składa się z mutowalnych zmiennych.
To podejście jest równoważne (w sensie działania monady) do przedstawionego poniżej.

\subsection{Ograniczenia techniczne Scali}

Do osiągnięcia mutowalnego stanu można było wykorzystać domknięcia, 
które składa się z mutowalnych
zmiennych. Funkcje leżące w domknięciu działają na mutowalnych zmiennych,
przez co można było zasymulować działanie stosu monad \mintinline{haskell}{ReaderT Refs IO},
gdzie \texttt{Refs} jest strukturą, która zawiera referencje do mutowalnych zmiennych.

To spotkało się z ograniczeniem nałożonym na wirtualną maszynę Javy. Według 
dokumentacji~\cite{Oracle:IC} zmienne w klasach anonimowych, bądź zagnieżdżonych, 
a stąd w domknięciach, które są modelowane jako klasy anonimowe w Scali, muszą zostać
nienaruszone przez funkcje operujące na tym domknięciu. 
Ze względu na to obostrzenie Scala tworzy kopię domknięcia, 
przez co mutowanie jest niemożliwe. 

To ograniczenie poprowadziło do fiaska
tego projektu, ponieważ jedyna skuteczna implementacja 
sprowadzała się do implementacji 
w sposób imperatywny.

\subsection{Elixir i niewłaściwy dobór paradygmatów}

Elixir jest językiem, którym celem jest ułatwienie pisania programów współbieżnie.
Oczywiście, współbieżność jest szczególnie wrażliwa na mutowalność, ponieważ jej
wprowadzenie generuje dodatkowy problem nadzorowania mutowalności poprzez muteksy,
bądź semafory. Elixir kładzie nacisk na model współbieżności, w którym mutowalność
jest kompletnie pominięta, stąd też wykorzystuje się model aktorów, który posługuje
się niemutowalnymi wiadomościami, które są wysyłane do aktorów.

\subsection{Dynamiczny system typów Elixira}

Przez to, że Elixir ma dynamiczny system typów, propozycja implementacji rozważana 
w drugim rozdziale jest nieadekwatna do systemu typów. Stąd należało porzucić
monady i skupiono się na implementacji kodowania za pomocą domknięć i mutowalnych 
zmiennych, w sposób nieznacznie odbiegający od programowania imperatywnego.

\subsection{Elixir i pozorna mutowalność zmiennych}

W przeciwieństwie do Erlanga, Elixir pozwala na redefinicję zmiennej, która de facto
jest etykietą. Wykonanie przypisania \texttt{x = 2} a następnie \texttt{x = 3}
usuwa z kontekstu przypisanie \texttt{x = 2}, by wprowadzić na nowo etykietę 
z wartością \texttt{x = 3}. Takie podejście nie pozwala na wykonanie 
domknięcia z mutowalnymi zmiennymi, co więcej, w żaden sposób nie pozwala
zamodelować mutowalnego stanu, który jest potrzebny w kodowaniu arytmetycznym, stąd
ten projekt również zakończył się niepowodzeniem.

\end{document}
