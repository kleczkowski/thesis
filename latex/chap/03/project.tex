\documentclass[../../praca.tex]{subfiles}

\begin{document}

\chapter{Projekt i implementacja kodera}

\section{Struktura projektu}

Projekt jest oparty o~system budowania \emph{Haskell Stack}.
Opis folderów znajdujących się w~projekcie oraz podział
na moduły znajduje się w~tabeli~\ref{tab:modules}.
Biblioteki i~motywacja ich~użycia są~podane w~tabeli~\ref{tab:libs}.
W~dolnej podtabeli zostały podane biblioteki użyte do~testowania
aplikacji.

Całość kodu jest zamieszczona na dołączonej płycie CD.
\begin{table}[h]
  \centering
  \subfloat[Struktura katalogowa]{\begin{tabular}{|l|p{6cm}|}
    \hline
    Nazwa & Opis \\ \hline
    \texttt{app}                & Zawiera kod odpowiadający za~interfejs tekstowy kodera. \\ \hline
    \texttt{bench}              & Zawiera kod testów obciążeniowych. \\ \hline
    \texttt{test}               & Zawiera kod testów jednostkowych. \\ \hline
    \texttt{src}                & Zawiera główny kod projektu. \\ \hline 
  \end{tabular}}
  \quad
  \subfloat[Moduły projektu]{\begin{tabular}{|l|p{6cm}|}
    \hline 
    Nazwa & Opis \\ \hline
    \texttt{Codec.AC.Internal}  & Zawiera wewnętrzny kod kodeka, który reeksportuje
                                  moduły w~folderze~\texttt{Internal}. Dokładniejszy
                                  ich~opis można znaleźć generując dokumentację.\\ \hline
    \texttt{Codec.AC.Decoder}   & Zawiera implementację dekodera 
                                  i~skojarzonych z~nim instancji klas typów. \\ \hline
    \texttt{Codec.AC.Encoder}   & Jak wyżej, tylko~że koder. \\ \hline
    \texttt{Codec.AC.FreqTree}  & Zawiera implementację drzewa Fenwicka w~oparciu 
                                  o~mutowalne tablice. \\ \hline
  \end{tabular}}
  \caption{Struktura katalogowa projektu \texttt{ac-haskell}} i moduły projektu \texttt{ac-haskell}
  \label{tab:modules}
\end{table}

\begin{table}[p]
  \centering
  \begin{tabular}{|l|p{8cm}|}
    \hline
    Nazwa & Opis \\ \hline
    \texttt{base}               & Biblioteka standardowa Haskella. \\ \hline
    \texttt{bytestring}         & Biblioteka służąca do~przetwarzania danych
                                  binarnych w~wygodnej i~efektywnej formie
                                  tablic opakowanych w~czysto funkcyjny interfejs. \\ \hline
    \texttt{fixed}              & Wydajna implementacja arytmetyki stałoprzecinkowej o~formacie 15.16. \\ \hline
    \texttt{pipes}              & Biblioteka gwarantująca potoki wraz z efektywną implementacją. \\ \hline
    \texttt{pipes-bytestring}   & Biblioteka-córka łącząca funkcjonalności obydwu bibliotek. \\ \hline
    \texttt{pipes-group}        & Biblioteka służąca do grupowania wyjścia potoków produkujących \\ \hline
    \texttt{mtl}                & Implementacja ważniejszych monad w~Haskellu. \\ \hline
    \texttt{optparse-applicative}&Biblioteka służąca do~parsowania argumentów podanych 
                                  do~interfejsu tekstowego kodera. \\ \hline
    \texttt{bytestring-arbitrary}&Biblioteka pozwalająca na~dostarczanie przykładów ciągów bajtów
                                  dla narzędzia testującego~\texttt{QuickCheck} \\ \hline
    \texttt{transformers}       & Biblioteka dostarczająca transformatory monad. \\ \hline
    \texttt{vector}             & Wydajna implementacja mutowalnych i~niemutowalnych tablic. \\ \hline
    \hline
    \texttt{criterion}          & Biblioteka służąca do~przeprowadzania testów obciążeniowych
                                  i~generacji raportów. \\ \hline
    \texttt{mwc-random}         & Generator pseudolosowy \emph{multiple and carry} służący 
                                  do~generowania danych losowych o~dość dobrym rozkładzie
                                  (dobrze oddający rozkład jednostajny). \\ \hline
    \texttt{QuickCheck}         & Biblioteka służąca do sprawdzania własności pewnych funkcji i~obiektów,
                                  która automatycznie generuje przypadki do~testowania, wraz z~podawaniem kontrprzykładów. \\ \hline
    \texttt{random}             & Zwyczajny generator pseudolosowy, używany 
                                  zazwyczaj w testach jednostkowych. \\ \hline
    \texttt{random-bytestring}  & Biblioteka pozwalająca na~generowanie losowych ciągów bajtów
                                  w~monadzie~\texttt{IO} na~potrzeby narzędzia~\texttt{criterion}. \\ \hline
  \end{tabular}
  \caption{Użyte biblioteki w~projekcie \texttt{ac-haskell}}
  \label{tab:libs}
\end{table}

\section{Konfiguracja projektu}

\emph{Haskell Stack} używa narzędzia o~nazwie \emph{hpack}. 
Jest to~generator plików narzędzia budowania~\emph{Cabal},
który jest tyłem dla~\emph{Stacka}. Sporządzenie konfiguracji
jest bardzo proste, a~sama~jej treść jest samoopisująca~się. 
Plik znajduje się jako listing~\ref{lst:package-yaml}.

% TODO: Wstawić tutaj listing z konfiguracją package.yaml

\begin{listing}
  \begin{minted}{yaml}
name:     ac-haskell
version:  0.1.0
synopsis: A streaming implementation of arithmetic coding
category: Compression
author:   Konrad Kleczkowski <konrad.kleczkowski@gmail.com>
license:  MIT

dependencies:
- base == 4.* && < 5.0
- bytestring
- pipes
- pipes-bytestring
- pipes-group
- fixed
- mtl
- lens
- transformers
- primitive
- vector

library:
  source-dirs:    src

executables:
  ac-haskell:
    source-dirs:  app
    main:         Main.hs
    dependencies:
    - ac-haskell
    - optparse-applicative

tests:
  ac-haskell-spec:
    source-dirs:  test
    main:         Spec.hs
    dependencies:
    - ac-haskell
    - hspec
    - QuickCheck
    - random
    - bytestring-arbitrary

benchmarks:
  ac-haskell-bench:
    source-dirs:  bench
    main:         Bench.hs
    dependencies:
    - ac-haskell
    - criterion
    - mwc-random
    - random-bytestring
  \end{minted}
  \caption{Przykładowy plik konfiguracyjny pakietu \texttt{ac-haskell}}
  \label{lst:package-yaml}
\end{listing}

Tak~sporządzony plik może posłużyć do~inicjalizacji projektu.
Należy uruchomić polecenie \texttt{stack init} w~folderze,
w~którym znajduje~się konfiguracja.

Po inicjalizacji Stack dobiera nam profil LTS (ang. \emph{long-term support})
i inicjalizuje wewnętrzną strukturę projektu. 

\subsection{Stack a Cabal}

Za podobne narzędzie uważa się również Haskell Cabal.

Cabal jest systemem budowania dla Haskella, gdy Stack jest pewnego rodzaju nakładką,
która rozwiązuje problem tzw. \emph{Cabal hell}.

Problemem, który można doświadczyć, gdy projekt jest oparty wyłącznie o Cabala,
jest podobny do \emph{DLL hell} i polega on na tym, że Cabal rozwiązując zależności
przy niedbałym doborze wersji bibliotek, może nie spełnić pozostałych wymagań
bibliotek. Przy wyborze zależności, nie można wybrać dwóch
wersji tej samej biblioteki, co może skutkować niespełnieniem wymogów przez Cabala.

Stąd powstała nakładka, która pozwala na pogrupowanie bibliotek w profile LTS,
które gwarantują nam kompatybilność bibliotek ze sobą. To podejście skutecznie rozwiązuje
problem \emph{Cabal hell}.

\section{Dokumentacja projektu}

Szczegółowa dokumentacja kodu jest generowana przez narzędzie \texttt{haddock}.
\texttt{Haddock} jest bogatym w funkcjonalności generatorem dokumentacji, 
analogicznym do narzędzi takich jak \texttt{javadoc} albo \texttt{Doxygen}, dla Haskella. 
Generator opiera się na bogatej składni
do formatowania tekstu, pozwala grupować nazwy w tematyczne grupy oraz
wprowadza makra do dokumentacji, dzięki którym można uniknąć powtarzalności podczas
pisania dokumentacji.

Aby wygenerować i~podejrzeć dokumentację projektu, należy uruchomić 
\mint{console}|stack haddock ac-haskell --open| w~folderze projektu. Po~zbudowaniu dokumentacji, pojawia~się
okno z~modułami projektu. Pod~linkami znajdują~się poszczególne dokumentacje
modułów wraz z~wyeksportowanymi symbolami. Do~każdego wyeksportowanego
symbolu pojawia się jego typ, instancje oraz donośnik do~kodu źródłowego,
gdzie można prześledzić implementację i~inne szczegóły.

Dokumentacja jest sporządzona w~języku angielskim.

\section{Instalacja programu}

Aby zainstalować program, należy uruchomić \texttt{stack install} 
w~folderze projektu. Wtedy binarka o~nazwie \texttt{ac-haskell}
zostaje skopiowana do~folderu \texttt{\~/.local/bin} (dla~Linuxa)
lub~\texttt{\%appdata\%/local/bin} (dla~Windowsa).

\section{Uruchomienie i~obsługa programu}

O~ile wyżej wymienione foldery znajdują się w~zmiennej środowiskowej
\texttt{PATH}, można wywołać program, by~podać wymaganą pomoc (w~języku angielskim),
za~pomocą polecenia~\texttt{ac-haskell --help}.

Aplikacja generuje odpowiednią pomoc dot.~poleceń. By~uzykać pomoc na~temat konkretnych
poleceń, wystarczy wywołać \texttt{ac-haskell <polecenie>}.

\begin{listing}
  \begin{minted}{console}
$ ac-haskell --help
ac-haskell --- arithmetic codec written in Haskell

Usage: ac-haskell COMMAND
  Compress data with the arithmetic codec

Available options:
  -h,--help                Show this help text

Available commands:
  encode                   Run arithmetic enoder
  decode                   Run arithmetic decoder
$ ac-haskell decode
Missing: (INPUT_FILE | --stdin) (OUTPUT_FILE | --stdout)

Usage: ac-haskell decode (INPUT_FILE | --stdin) (OUTPUT_FILE | --stdout)
  Run arithmetic decoder
  \end{minted}
  \caption{Przykładowe użycie pomocy programu \texttt{ac-haskell}}
\end{listing}

\section{Testy projektu}

Projekt zawiera dwa~rodzaje testów: testy jednostkowe, sprawdzające
czy~kod spełnia wymagania, oraz obciążeniowe, które sprawdzają
efektywność rozwiązania.

\subsection{Testy jednostkowe}

Testy jednostkowe odpowiadają za sprawdzenie, czy aplikacja zachowuje
się zgodnie ze specyfikacją. Sprawdzane jest, czy dekodowanie odwraca
proces kodowania. Testy zostały w pełni zautomatyzowane, dzięki
użytej bibliotece QuickCheck. Biblioteka pozwala na generowanie przypadków
testowych wraz z kontrolą niepowodzeń w trakcie testów. Jeśli podczas testu
nadarzy się błąd, środowisko testowe przycina przypadki do mniejszych, tak by
znaleźć odpowiednio mały kontrprzykład do testowanych własności kodu. 
Takie podejście pozwala na skuteczniejsze testowanie kodu bez interakcji w tworzenie
przypadków testowych oraz skraca czas pisania testów~\cite{Claessen:QLT}.

QuickCheck wspiera tzw. \emph{property-based testing}, skupiając się 
na własnościach algebraicznych funkcji, które są testowane.

By~uruchomić testy jednostkowe, należy wywołać~\texttt{stack test}
w~folderze projektu. Narzędzie \emph{Stack} zbuduje projekt
i~stowarzyszone z~nim testy. Ponadto zbuduje~kod
odpowiedzialny za~sprawdzanie przykładów (wcześniej
wymieniony~\texttt{doctest}). Testy zostaną automatycznie
znalezione i~uruchomione. \texttt{hspec} 
generuje raport z~przeprowadzonych testów jednostkowych.

\subsection{Testy obciążeniowe}

Testy obciążeniowe pozwalają na pomiar czasu wykonywania implementacji projektu.
Biblioteka \texttt{criterion} pozwala na osadzenie testów oraz generowanie
odpowiedniego raportu, wraz z danymi statystycznymi, osadzonego w pliku tekstowym,
bądź pliku HTML.

By~uruchomić testy obciążeniowe, należy uruchomić~\texttt{stack bench}
w~folderze projektu. \emph{Stack} zbuduje projekt
i~dodatkowy~kod, którego zadaniem jest zmierzenie
kosztownych operacji kodowania i~dekodowania,
w~różnych scenariuszach, o~których będzie mowa
w~następnym rozdziale. Również zostaną uruchomione
testy i~zostanie wygenerowany tekstowy raport. 
By~otrzymać wersję HTML raportu, należy wywołać
\mint{console}|stack bench --benchmark-arguments '--output-file raport.html'|

\end{document}
