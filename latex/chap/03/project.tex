\documentclass[../../thesis.tex]{subfiles}

\begin{document}

\chapter{Projekt kodera}

\section{Struktura projektu}

Projekt jest oparty o~system budowania \emph{Haskell Stack}.
Opis folderów znajdujących się w~projekcie znajduje się w~tabeli~\ref{tab:directories}.
Opis modułów wchodzących w~skład projektu są~w~tabeli~\ref{tab:modules}.
Biblioteki i~motywacja ich~użycia są~podane w~tabeli~\ref{tab:libs}.
W~dolnej podtabeli zostały podane biblioteki użyte do~testowania
aplikacji.

\begin{table}
  \label{tab:directories}
  \caption{Struktura katalogowa projektu \texttt{ac-haskell}}
  \centering
  \begin{tabular}{|l|p{8cm}|}
    \hline
    Nazwa & Opis \\ \hline
    \texttt{app}                & Zawiera kod odpowiadający za~interfejs tekstowy kodera. \\ \hline
    \texttt{bench}              & Zawiera kod testów obciążeniowych. \\ \hline
    \texttt{test}               & Zawiera kod testów jednostkowych. \\ \hline
    \texttt{src}                & Zawiera główny kod projektu. \\ \hline 
  \end{tabular}
\end{table}

\begin{table}
  \label{tab:modules}
  \caption{Moduły projektu \texttt{ac-haskell}}
  \centering
  \begin{tabular}{|l|p{8cm}|}
    \hline 
    Nazwa & Opis \\ \hline
    \texttt{Codec.AC.Internal}  & Zawiera wewnętrzny kod kodeka, który reeksportuje
                                  moduły w~folderze~\texttt{Internal}. Dokładniejszy
                                  ich~opis można znaleźć generując dokumentację.\\ \hline
    \texttt{Codec.AC.Decoder}   & Zawiera implementację dekodera 
                                  i~skojarzonych z~nim instancji klas typów. \\ \hline
    \texttt{Codec.AC.Encoder}   & Jak wyżej, tylko~że koder. \\ \hline
    \texttt{Codec.AC.FreqTree}  & Zawiera implementację drzewa Fenwicka w~oparciu 
                                  o~mutowalne tablice. \\ \hline
  \end{tabular}
\end{table}

\begin{table}
  \label{tab:libs}
  \caption{Użyte biblioteki w~projekcie \texttt{ac-haskell}}
  \centering
  \begin{tabular}{|l|p{8cm}|}
    \hline
    Nazwa & Opis \\ \hline
    \texttt{base}               & Biblioteka standardowa Haskella. \\ \hline
    \texttt{bytestring}         & Biblioteka służąca do~przetwarzania danych
                                  binarnych w~wygodnej i~efektywnej formie
                                  tablic opakowanych w~czysto funkcyjny interfejs. \\ \hline
    \texttt{fixed}              & Wydajna implementacja arytmetyki stałoprzecinkowej o~formacie 15.16. \\ \hline
    \texttt{pipes}              & Biblioteka gwarantująca potoki wraz z efektywną implementacją. \\ \hline
    \texttt{pipes-bytestring}   & Biblioteka-córka łącząca funkcjonalności obydwu bibliotek. \\ \hline
    \texttt{pipes-group}        & Biblioteka służąca do grupowania wyjścia potoków produkujących \\ \hline
    \texttt{mtl}                & Implementacja ważniejszych monad w~Haskellu. \\ \hline
    \texttt{optparse-applicative}&Biblioteka służąca do~parsowania argumentów podanych 
                                  do~interfejsu tekstowego kodera. \\ \hline
    \texttt{bytestring-arbitrary}&Biblioteka pozwalająca na~dostarczanie przykładów ciągów bajtów
                                  dla narzędzia testującego~\textt{QuickCheck} \\ \hline
    \texttt{transformers}       & Biblioteka dostarczająca transformatory monad. \\ \hline
    \texttt{vector}             & Wydajna implementacja mutowalnych i~niemutowalnych tablic. \\ \hline
    \hline
    \texttt{criterion}          & Biblioteka służąca do~przeprowadzania testów obciążeniowych
                                  i~generacji raportów. \\ \hline
    \texttt{mwc-random}         & Generator pseudolosowy \emph{multiple and carry} służący 
                                  do~generowania danych losowych o~dość dobrym rozkładzie
                                  (dobrze oddający rozkład jednostajny). \\ \hline
    \texttt{QuickCheck}         & Biblioteka służąca do sprawdzania własności pewnych funkcji i~obiektów,
                                  która automatycznie generuje przypadki do~testowania, wraz z~podawaniem kontrprzykładów. \\ \hline
    \texttt{random}             & Zwyczajny generator pseudolosowy, używany 
                                  zazwyczaj w testach jednostkowych. \\ \hline
    \texttt{random-bytestring}  & Biblioteka pozwalająca na~generowanie losowych ciągów bajtów
                                  w~monadzie~\texttt{IO} na~potrzeby narzędzia~\texttt{criterion}. \\ \hline
  \end{tabular}
\end{table}

\section{Konfiguracja projektu}

\emph{Haskell Stack} używa narzędzia o~nazwie \emph{hpack}. 
Jest to~generator plików narzędzia budowania~\emph{Cabal},
który jest tyłem dla~\emph{Stacka}. Sporządzenie konfiguracji
jest bardzo proste, a~sama~jej treść jest samoopisująca~się.

% TODO: Wstawić tutaj listing z konfiguracją package.yaml

Tak~sporządzony plik może posłużyć do~inicjalizacji projektu.
Należy uruchomić polecenie \texttt{stack init} w~folderze,
w~którym znajduje~się konfiguracja.

\section{Dokumentacja projektu}

Szczegółowa dokumentacja kodu jest generowana przez narzędzie \emph{haddock}.
Aby wygenerować i~podejrzeć dokumentację projektu, należy uruchomić \texttt{stack haddock
ac-haskell --open} w~folderze projektu. Po~zbudowaniu dokumentacji, pojawia~się
okno z~modułami projektu. Pod~linkami znajdują~się poszczególne dokumentacje
modułów wraz z~wyeksportowanymi symbolami. Do~każdego wyeksportowanego
symbolu pojawia się jego typ, instancje oraz donośnik do~kodu źródłowego,
gdzie można prześledzić implementację i~inne szczegóły.

Dokumentacja jest sporządzona w~języku angielskim.

\section{Instalacja programu}

Aby zainstalować program, należy uruchomić \texttt{stack install} 
w~folderze projektu. Wtedy binarka o~nazwie \texttt{ac-haskell}
zostaje skopiowana do~folderu \texttt{\~/.local/bin} (dla~Linuxa)
lub~\texttt{tutaj/sprawdz/windows} (dla~Windowsa).

\section{Uruchomienie i~obsługa programu}

O~ile wyżej wymienione foldery znajdują się w~zmiennej środowiskowej
\texttt{PATH}, można wywołać program, by~podać wymaganą pomoc (w~języku angielskim),
za~pomocą polecenia~\texttt{ac-haskell --help}.

Aplikacja generuje odpowiednią pomoc dot.~poleceń. By~uzykać pomoc na~temat konkretnych
poleceń, wystarczy wywołać \texttt{ac-haskell <polecenie>}.

\section{Testy projektu}

Projekt zawiera dwa~rodzaje testów: testy jednostkowe, sprawdzające
czy~kod spełnia wymagania, oraz obciążeniowe, które sprawdzają
efektywność rozwiązania.

\subsection{Testy jednostkowe}

By~uruchomić testy jednostkowe, należy wywołać~\texttt{stack test}
w~folderze projektu. Narzędzie \emph{Stack} zbuduje projekt
i~stowarzyszone z~nim testy. Ponadto zbuduje~kod
odpowiedzialny za~sprawdzanie przykładów (wcześniej
wymieniony~\texttt{doctest}). Testy zostaną automatycznie
znalezione i~uruchomione. \texttt{hspec} 
generuje raport z~przeprowadzonych testów jednostkowych.

\subsection{Testy obciążeniowe}

By~uruchomić testy obciążeniowe, należy uruchomić~\texttt{stack bench}
w~folderze projektu. \emph{Stack} zbuduje projekt
i~dodatkowy~kod, którego zadaniem jest zmierzenie
kosztownych operacji kodowania i~dekodowania,
w~różnych scenariuszach, o~których będzie mowa
w~następnym rozdziale. Również zostaną uruchomione
testy i~zostanie wygenerowany tekstowy raport. 
By~otrzymać wersję HTML raportu, należy wywołać
\texttt{stack bench --benchmark-arguments '--output-file raport.html'}.

\end{document}
