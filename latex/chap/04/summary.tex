\documentclass[../../thesis.tex]{subfiles}

\begin{document}

\chapter{Podsumowanie}

\section{Struktura testów obciążeniowych}

Do testów zaimplementowanego kodera, jako próbkę referencyjną
wykorzystano koder arytmetyczny autorstwa \emph{Project Nayuki}
pod licencją MIT. Jego kod źródłowy jest dostępny pod linkiem
\verb|https://github.com/nayuki/Reference-arithmetic-coding|.

Testy składają się z serii pięciu wykonań nad różnymi przypadkami
testowymi. Spośród wyników wykluczany jest najgorszy i najlepszy
wynik, a z pozostałych trzech wyciągana jest średnia arytmetyczna.

Sporządzono następujące przypadki:
\begin{itemize}
  \item \texttt{pan-tadeusz.txt} --- plik tekstowy zawierający
    ,,Pana Tadeusza'' Adama Mickiewicza jako przypadek, w którym
    testowany jest przypadek średni. Plik pochodzi z Wolnych Lektur,
    jego tekst podlega domenie publicznej, a przypisy i inne
    prace związane z tekstem są powiązane licencją CC-BY-SA 3.0;
  \item \texttt{zeros.1m} --- jednomegabajtowy plik zawierający
    wyłącznie zera, przypadek optymistyczny;
  \item \texttt{random.1m} --- jednomegabajtowy plik zawierający 
    losowe bajty danych pochodzące z \texttt{/dev/random}, który
    reprezentuje kryptograficznie bezpieczny generator pseudolosowy
    o rozkładzie bardzo zbliżonym do jednostajnego, przypadek
    pesymistyczny.
\end{itemize}

\section{Wyniki testów obciążeniowych}



\section{Wnioski}

\end{document}
